\usepackage{fontspec}
\usepackage[T1]{fontenc}	% Pour l'encodage des caractères, nécessaire au lieu de fontspec pour utiliser les kp-fonts
\usepackage[light,oldstyle,fulloldstylenums,nott,easyscsl]{kpfonts}	% Permet d'utiliser les kp-fonts
\usepackage[utf8x]{inputenc}
\usepackage{xltxtra}
 \DeclareTextCommandDefault{\nobreakspace}{\leavevmode\nobreak\ } 
\usepackage[english,francais]{babel}
\usepackage[protrusion=true]{microtype}
\usepackage[threshold=3]{csquotes}
	% for commands "\blockquote" and related:
	\renewcommand{\mkblockquote}[4]{\medskip\begin{spacing}{1.25}\openautoquote\small#1\closeautoquote#2#4#3\end{spacing}\medskip}
	\renewcommand{\mkcitation}[1]{#1}
	% for environments "displayquote" and related:
	\renewcommand{\mkbegdispquote}[2]{\openautoquote}
	\renewcommand{\mkenddispquote}[2]{\closeautoquote#1#2}

\usepackage{setspace}	% permet de modifier l'interligne du document
\usepackage[top=2cm,left=2cm,right=2cm,bottom=2.25cm,a4paper]{geometry}  % permet de modifier les marges, etc.
	\widowpenalty=10000		% pour la gestion des veuves et des orphelines
	\clubpenalty=10000
\usepackage{lettrine}
\usepackage[bookmarks=true,hyperindex=true,colorlinks=false,linktoc=all,plainpages=false]{hyperref}	% permet des hyperliens cliquables
	\urlstyle{same}
\usepackage{array}  % permet un meilleur affichage des tableaux
	\usepackage{multirow}
\usepackage{amsmath}
\usepackage{eurosym}    % permet d'avoir un symbole € avec la commande \EUR{}
\usepackage{fontawesome}
\usepackage{tabularx}
\usepackage{booktabs}
\usepackage{caption}	% Offre une meilleure gestion des légendes
	\captionsetup[table]{name=Tableau, labelsep=endash,justification=centering,textfont={it,footnotesize},labelfont={small}, width=.8\textwidth}
	\captionsetup[figure]{name=Figure, labelsep=endash,justification=centering,textfont={it,footnotesize},labelfont={small}, width=.8\textwidth}
	\captionsetup[subfloat]{justification=centering,textfont={it,scriptsize},labelfont={footnotesize}, width=.4\textwidth}
\usepackage[lofdepth, lotdepth, font={rm,md,sl,footnotesize}]{subfig} % permet d'avoir des figures subdivisées
\usepackage{framed} % permet la bordure gauche de certains paragraphes
\usepackage{fancybox}
\usepackage{wrapfig}	% pour mettre des illustrations enveloppees de texte
\usepackage{fmtcount} % permet d'afficher un compteur sous plusieurs formes
\usepackage{paralist}	% permet l'obtention de listes inline
\usepackage{numprint}	% permet d'avoir une espace entre chaque milliers
\usepackage[nice]{nicefrac}		% permet de réaliser des fractions pour du texte avec la commande \nicefrac{}{} (e.g. 1/4)
\usepackage{paralist}	% permet l'obtention de listes inline
\usepackage{graphicx}
\usepackage[svgnames]{xcolor}
\usepackage{tikz}
	\usetikzlibrary{shapes.multipart,shapes.misc,decorations.text,patterns,decorations.pathreplacing}
\usepackage{color}		% permet d'utiliser les couleurs
	\definecolor{rouge}{rgb}{0.75,0,0}
	\definecolor{gris}{rgb}{.50,.50,.50}
	\definecolor{grisclair}{rgb}{.25,.25,.25}
\usepackage[isbn=false, sorting=nyt, hyperref=true, url=false, maxnames=4, minnames=1, bibstyle=/Users/thenri/Documents/10_scolarite/10_universite/m1_meef/mabiblio, citestyle=verbose]{biblatex}	% pour la gestion de la bibiliographie
%	\bibliography{biblio}
\usepackage{multicol}
	\setlength{\columnsep}{2em}
	\setlength{\columnseprule}{0pt}
\usepackage{imakeidx}	% pour la creation d'index
	\makeindex

\setcounter{tocdepth}{5}

\usepackage{fancyhdr}
	\pagestyle{fancy}
	\fancyhf{}
	\fancyfoot[LE, RO]{\thepage{}}
	\renewcommand{\headrulewidth}{0pt}
	\renewcommand{\footrulewidth}{0pt}

\input{/Users/thenri/Documents/20_informatique/latex/30_packages_perso/namedisp/namedisp.sty}